% Ubah kode buku berikut dengan yang ditentukan oleh departemen
\begin{large}
  FINAL PROJECT PROPOSAL - TD123456
\end{large}

\vspace{\fill}

% Ubah kalimat berikut dengan judul tugas akhir
\begin{spacing}{1.5}
  \begin{Large}
    UTILIZATION OF GENETIC ALGORITHMS FOR THE 
    SCHEDULING PROCESS OF LECTURES IN THE 
    DEPARTMENT OF COMPUTER ENGINEERING ITS
  \end{Large}
\end{spacing}

\vspace{\fill}

% Ubah kalimat-kalimat berikut dengan nama dan NRP mahasiswa
\begin{large}
  Muhammad Zakariya Nur Ramdhani \\
  \textmd{NRP 0721 19 4000 0016}
\end{large}

\vspace{\fill}

% Ubah kalimat-kalimat berikut dengan nama-nama dosen pembimbing
\begin{large}
  \textmd{Advisor} \\
  Dr.\ Diah Puspito Wulandari, S.T., M.Sc\\
  \textmd{NIP 19801219 200501 2 000} \\
  Dr.\ Supeno Mardi Susiki Nugroho, ST., MT. \\
  \textmd{NIP 19700313 199512 1 001}
\end{large}

\vspace{\fill}

% Ubah kalimat-kalimat berikut dengan nama departemen dan fakultas
Study Program of Multimedia and IoT \\

\mdseries

Department of Computer Engineering \\
Faculty of Intelligent Electrical and Informatics Technology\\
Sepuluh Nopember Institute of Technology

% Ubah kalimat berikut dengan tempat dan tahun pembuatan buku
Surabaya \\
2022

