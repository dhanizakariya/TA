\section{TINJAUAN PUSTAKA}

% Ubah konten-konten berikut sesuai dengan isi dari tinjauan pustaka
\subsection{Hasil penelitian/perancangan terdahulu}
\lipsum[3]

\subsection{Teori/Konsep Dasar}

\subsubsection{Hukum Newton}

% Contoh penggunaan referensi dari pustaka
Newton pernah merumuskan \parencite{Newton1687} bahwa \lipsum[8]
% Contoh penggunaan referensi dari persamaan
Kemudian menjadi persamaan seperti pada persamaan \ref{eq:FirstLaw}.

% Contoh pembuatan persamaan
\begin{equation}
  % Label referensi dari persamaan yang dibuat
  \label{eq:FirstLaw}
  % Baris kode persamaan yang dibuat
  \sum \mathbf{F} = 0\; \Leftrightarrow\; \frac{\mathrm{d} \mathbf{v} }{\mathrm{d}t} = 0.
\end{equation}

\lipsum[9]

\subsubsection{Anti Gravitasi}

\lipsum[10]
